\documentclass[12pt]{exam}
\usepackage{jde_course_latex}
\usepackage{arydshln}
\usepackage{titlesec}

\titleformat{\section}
  {\Large\fontfamily{cmr}\selectfont}{\thesection}{1em}{}

\firstpageheadrule
\firstpagefootrule
\runningheadrule
\runningfootrule
\lhead{Integrated MSE Computational Models --- Roadmap}
\rhead{Summer 2019}
\lfoot{\includegraphics[height=0.75cm]{./Figures/CT-STEM.pdf}}
\cfoot{}
\rfoot{\thepage}

%Begin Document-----------------------------------------
\begin{document}
			
{\Large \textcolor{NUpurp120}{Integrated MSE Computational Models --- Roadmap}}

The \textit{Overleaf} source file for this assignment can be found \href{https://www.overleaf.com/project/5d38641245f8bc723f3e7708}{here}.
\vspace{2em}

Let's figure out why opportunities lay for implementation of CMS into MSE core courses at a modular level.

\newpage
\section{\fontfamily{cmr}\selectfont MAT\_SCI 201-301}

\begin{itemize}
    \item Lennard-Jones models and the use of pair-potentials in molecular dynamics simulations.
    \item Hume-Rothery predictions \textit{via} data-mining.
    \item Polymer statistics --- statistical model of polymer end-to-end distance.
\end{itemize}

\newpage
\section{MAT\_SCI 314}

\newpage
\section{MAT\_SCI 315}

\newpage
\section{MAT\_SCI 316-1}

\newpage
\section{MAT\_SCI 316-2}

\newpage
\section{MAT\_SCI 351-2}

\newpage
\section{MAT\_SCI 391}

\end{document}
