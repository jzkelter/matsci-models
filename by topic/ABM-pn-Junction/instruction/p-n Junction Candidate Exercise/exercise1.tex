\documentclass[12pt]{exam}
\usepackage{jde_course_latex}
\usepackage{mathrsfs} 
\usepackage{arydshln}
\usepackage[
backend=biber,
style=alphabetic,
sorting=ynt]
{biblatex}
\addbibresource{references.bib}

\firstpageheadrule
\firstpagefootrule
\runningheadrule
\runningfootrule
\lhead{Agent-Based Modeling --- The $p$-$n$ Junction}
\rhead{Draft --- December 2019}
\lfoot{\includegraphics[height=0.75cm]{./Figures/CT-STEM.pdf}}
\cfoot{}
\rfoot{\thepage}

 %Flip flag between \noprint* and \print to toggle environment display.
\printanswers
\printlabguide
\noprintID
\noprintrubric
\noprintoutcomes
\noprintcomments

%Solution typestting------------------------
\unframedsolutions
\pointsinrightmargin
\CorrectChoiceEmphasis{\color{red}\bfseries}
\SolutionEmphasis{\color{red}}

%Begin document-----------------------------
\begin{document}
			
{\Large \textcolor{NUpurp120}{MAT\_SCI 351-2: The $p$-$n$ Junction}}

%Collaboration Guide----------------------------------
\begin{LabGuide}

The \textit{Overleaf} project for this assignment can be found  at:

\begin{center}
\href{https://www.overleaf.com/project/https://www.overleaf.com/read/qtyzpkccphrv}{https://www.overleaf.com/read/qtyzpkccphrv}.
\end{center}

Contact Jonathan Emery at

\begin{center}
\href{mailto:jonathan.emery@northwestern.edu}{jonathan.emery@northwestern.edu}
\end{center}

to acquire collaborative write permissions for this project.

\end{LabGuide}

%Exercise-------------------------------
The $p$-$n$ junction is a central building block to critical electronic devices such as diodes, transistors, light-emitting diodes (LEDs) and photovoltaic (PV) cells. In this exercise, we will explore $p$-$n$ junction behavior using a molecular-dynamics classical charge carrier transport model in \textsc{NetLogo}. At the end of this assignment you will be able to 

\begin{enumerate}
    \item Describe and interpret the electronic phenomena that occur when when $p$-type semiconductor is in contact with an $n$-type semiconductor.
    \item Derive the important $p$-$n$ junction properties ($p$, $n$, $\rho_{\mathrm{net}}$, $\mathcal{E}_x$, $V(x)$, and $W$) that emerge due to carrier diffusion at the interface.
    \item Adapt various materials-related parameters (band gap, charge mobility, dopant concentration, etc.) to deduce the influence of materials properties on device behavior.
\end{enumerate}

\section*{Model Description} %-NA - J. Kelter Edit.

We'll use a computational tool constructed in \textsc{NetLogo} by J. Kelter called \texttt{p-n-junction}. This simulation models a $p$-$n$ junction using molecular dynamics. The model consists of both intrinsic and extrinsic charge carriers that accelerate classically under an applied field. For a more complete description of the model, refer to the \texttt{Info} tab in the model. \textbf{The user can explore the effects of the following factors on fields and potentials, charge carrier distributions, and currents:}

\begin{enumerate}
    \item The influence of the initial number of $p$ or $n$ dopants.
    \item Thermal activation of carriers is controlled using the band gap and temperature.
    \item The likelihood of scattering (random changes in velocity at some time step).
    \item The effect of an applied voltage on the behavior.
\end{enumerate}

\newpage


%Begin Questions--------------------------------
\begin{questions}
\question \textcolor{NUpurp120}{\Large\sffamily{Properties of the $p$-$n$ Junction}} 

We'll begin by familiarizing ourselves with the simulation tool. Here, we'll explore the changes in the model's outputs by varying input parameters. First, however, you need access the simulation tool
    
    \begin{enumerate}
        \item Download an install the \textsc{NetLogo} 6.1.1.\href{https://ccl.northwestern.edu/netlogo/6.1.1/}{https://ccl.northwestern.edu/netlogo/6.1.1/}  desktop application. Get the correct version for your operating system.
        \item Download and open the \textsc{NetLogo} model \texttt{p-n-junction.nlogo} from \textit{Canvas}.
    \end{enumerate} 
    
Once you've loaded the model, record the default settings and press ``setup''.

\begin{parts}
    \part Observe the physical representation of the model.
\end{parts}


\end{questions}


%FileID--------------------------------
\begin{FileID}
	\begin{center}
		\begin{tabular}{ll}
			\hline
			\hline
			\#FileTag:351.2-Devices-ABM-E1-1.tex & \#SourceTag:Su19-Original\\
			\#AuthorTag:JDEmery          & \#AuthorTag:JKelter \\
			\#UseTag:In-class            & \#UseTag:Homework \\
			\#AlignmentTag:Kasap3-Ch6 (\cite{Kasap:2005:PEM:1594045})    & \#AlignmentTag: \\
			\#TopicTag:Semiconductor     & \#TopicTag:pnJunction\\
			\#TopicTag:Computational     & \#TopicTag:ShortAnswer \\
			\#TaxonomyTag:Analyze        & \#TaxonomyTag:Create \\
			\hline
		\end{tabular}
	\end{center}
\end{FileID}

%Rubric--------------------------------
\begin{rubric}

\begin{enumerate}
	\item 
\end{enumerate}

\end{rubric}

%Outcomes--------------------------------
\begin{outcomes}
	\begin{center}
		\begin{tabular}{cccc}
			\hline\hline
			Class-Term & Instructor & Assessment & Results (Full/Partial/No Credit|Ave)\\
			\hline
			201-W20 & Emery-Lauhon & In-class & -\%/ -\%/ -\%|-\%\\
			\hline
		\end{tabular}
	\end{center}
\end{outcomes}

%Comments--------------------------------
\begin{comments}

\begin{itemize}
    \item How can we integrate (or touch on the physical implications of) the continuity equation or the Poisson equation? See \href{https://www.tf.uni-kiel.de/matwis/amat/semi_en/index.html}{here}?
\end{itemize}

\end{comments}

\newpage
\printbibliography


\end{document}